\documentclass{tufte-handout}

\title{How to Read a Philosophical Essay\\
	(Philosophy of Sport)}

\author[Craig T. Carley]{Professor Craig Carley}

\date{Phoenix College: Spring 2014} % without \date command, current date is supplied

%\geometry{showframe} % display margins for debugging page layout

\usepackage{graphicx} % allow embedded images
  \setkeys{Gin}{width=\linewidth,totalheight=\textheight,keepaspectratio}
  \graphicspath{{graphics/}} % set of paths to search for images
\usepackage{amsmath}  % extended mathematics
\usepackage{booktabs} % book-quality tables
\usepackage{units}    % non-stacked fractions and better unit spacing
\usepackage{multicol} % multiple column layout facilities
\usepackage{lipsum}   % filler text
\usepackage{fancyvrb} % extended verbatim environments
  \fvset{fontsize=\normalsize}% default font size for fancy-verbatim environments
\usepackage{colortbl}
\usepackage{hyperref}

% Standardize command font styles and environments
\newcommand{\doccmd}[1]{\texttt{\textbackslash#1}}% command name -- adds backslash automatically
\newcommand{\docopt}[1]{\ensuremath{\langle}\textrm{\textit{#1}}\ensuremath{\rangle}}% optional command argument
\newcommand{\docarg}[1]{\textrm{\textit{#1}}}% (required) command argument
\newcommand{\docenv}[1]{\textsf{#1}}% environment name
\newcommand{\docpkg}[1]{\texttt{#1}}% package name
\newcommand{\doccls}[1]{\texttt{#1}}% document class name
\newcommand{\docclsopt}[1]{\texttt{#1}}% document class option name
\newenvironment{docspec}{\begin{quote}\noindent}{\end{quote}}% command specification environment

\begin{document}

\maketitle % this prints the handout title, author, and date

%------------------------------------ DEFINE COLORS -------------------
\definecolor{blue}{HTML}{ADD8E6}
\definecolor{lightblue}{HTML}{D4E1E6}
\definecolor{darkblue}{HTML}{172285}
\definecolor{red}{HTML}{E69AA2}
\definecolor{lightred}{HTML}{E6C4C6}
\definecolor{darkred}{HTML}{9E0B0B}
\definecolor{brown}{HTML}{665322}


\section{Tricky Triad: Games, Play, and Sport}
\subsection{by Bernard Suits}

Since I have already published individual articles on games, play, and sport, one might ask why I am coming around again to peddle commodities that have already been merchandised.\sidenote{This paper was presented as the invited Annual Homecoming Lecture of the Faculty of Physical Education of the University of Western Ontario in London, Canada, October 14,1988} There are two reasons: (a) I have changed my views in some important ways about play and sport (though not about games), and (b) I am more interested in this inquiry in \textsc{\color{darkred}{relations}} among the three than I am in \textsc{\color{darkred}{distinctions}} between them.\sidenote{\color{darkred}What sorts of things are relations and distinctions?} 

Let me begin by making a large cut, as it were, between play on the one hand and games and sports on the other. Games and sports are enterprises or institutions, I suggest, in which the exhibition of skill is the paramount consideration, and I am not going to argue that in play the exhibition of skill is not the paramount consideration. Actually I am not going to argue quite that. For an objection immediately arises. Are not games, it will be asked, \textit{play} when they are undertaken as ends in themselves, by amateurs, that is, by those whose gaming is for the love of it, in contrast to professionals, whose gaming is for sometimes astronomical salaries? If so, then the difference between amateur and professional gamespersons would appear to be precisely the difference between play and work. And since in amateur gaming skill \textit{is} the paramount consideration, it would seem false to say of play, or at least to say of all play, that the exhibition of skill is \textit{not} of paramount importance. 

This is a serious objection and can be met, if I am to maintain my large cut between play and games/sport by making a distinction between two kinds of activity that are normally called play. It is the distinction between what I shall call primitive and what I shall call sophisticated play. \color{darkblue}Primitive play\color{Black} is play and only play; \color{darkblue}sophisticated play \color{Black} is play and something else as well.

\subsection{Primitive Play}

What do I mean by primitive play, such that skill is not its essential ingredient? Well, let us see if we can find a \textsc{paradigm}\sidenote{What is a paradigm?}, or at least an example that will not raise more questions than it answers. How about this one? The baby playing with the water in its bath. Now, I think we can agree that the baby is playing for the following \textit{negative} reason: it is not working, that is, it is not engaged in any \textsc{instrumental}\sidenote{What is an instrumental enterprise?} enterprise. More to the point, it is not engaged in bathing itself nor is it assisting its father or mother in being bathed, for the obvious reason that it does not know what it is doing in the water at all. Very well: the baby is not working. But is it doing \textit{anything}? Or is it simply behaving randomly, as a robot whose circuitry is defective might exhibit bits of behavior unrelated to each other and to no \textsc{purpose}? I think not. The baby is not acting randomly or without purpose because it is clear that when the baby plays with the water in its bath there is input on its part---it splashes the water. And there is feedback---it is pleased by its water-splashing; and there is, as a consequence, further input: the baby continues to splash to get further feedback. 

Now, it will be pointed out that the more the baby engages in this activity, during this bath and in subsequent baths, the more skillful it will become at splashing, or at least at splashing in ways that pay off. Skills, it will be said, will begin to emerge in the baby's behavior. No doubt. I readily grant that. But, and this is the main point I want to make about play and skill, \textit{the skills learned are not the payoff the baby is seeking}. The baby is not pleased by the fact that it is becoming an accomplished water-splasher; it is pleased by the splashing water. 

Primitive play, I suggest, is not concerned primarily with the exercise and enjoyment of skills but with the introduction of new experiences that arise, usually, serendipitously. Still, the repetition of these experiences may very well result in the development of skills directed toward the recurrence of those experiences, and such skills may, although they need not, come to be \textsc{valued for their own sake}. When that \textit{does} happen we are just beginning to move from primitive play to sophisticated play, that is, to games, and perhaps to something else as well. 

\subsection{Sport}

I ask you to keep the foregoing at the back of your mind for the present, for we shall be returning to it. But now I would like to malt: a head-on attack on sports. This will lead us back. in due course, to the distinction between primitive and sophisticated play, and to an additional, and parallel, distinction. 

Although other things are, and are called, sport. I shall confine my remarks this afternoon to sport as the kind of activity exemplified in the Olympic Games. Sports so understood may be defined. or at least described, as competitive events involving a variety of physical (usually in combination with other) human skills, where the superior participant is judged to have exhibited those skills in a superior way. What we are talking about, then. are competitive athletic events. May we say. then, that such sports are the same as athletic \textit{games}? No, we may not. In an article of mine recently reprinted in an anthology on the philosophy of sport. l maintained that sports of the kind we are now considering were the same as athletic games.\sidenote{The article in question is "The Elements of Sport," reprinted in \textit{Philosophic Inquiry in Sport} Edited by William J. Morgan and Klaus V. Meier, Champaign, IL: Human Kinetics, 1988} Well, I was wrong. The Olympics (as well as the Commonwealth Games, and so on) contain two distinct types of competitive event, what I have elsewhere called judged as opposed to refereed events. One is a performance and so requires judges. The other is not a performance but a rule-governed interplay of participants, and so requires not judges but law-enforcement officers. that is. referees Performances require rehearsal, games require practice. To be sure, football teams. in a sense. do rehearse certain moves or plays. but that is simply a part of practice, and victory is not determined by the artistry of such plays or moves but by their effectiveness in winning games. But I submit that diving and gymnastic competitions are no more games than are other judged competitive events such as beauty contests and pie-baking competitions. 

\subsection{Sport and Skills}

If you agree with me that the sports of the Olympics are of two quite different kinds, then perhaps you will also agree with me that the skills required by each are also of two quite different kinds. If, as I suggested earlier, play becomes transformed into game when the skills learned in serendipitous play, instead of being instrumental to other payoffs (such as the splashing of bath water) themselves constitute the payoff, then we might consider the possibility that \textit{performances}, in contrast to \textit{games}, arise not out of the increased sophistication of play but out of an increased sophistication of work---or at least of certain serious endeavors. 

What possible reason could I have for making this suggestion? Actually I have two reasons. The first is that such a hypothesis presents us with an attractive symmetry in distinguishing games that are sports from performances that are sports. Consider the following example. A boy is walking down the street. He sees a tin can. He idly kicks it and enjoys the sight of it tumbling along the street, and he enjoys the sound of its ear-shattering rattle. Another boy approaches from the opposite direction. As the can reaches him. he kicks it back. We have here the serendipitous origin of Kick the Can. field hockey, and a number of other games as well, as goal lines are added. and then perhaps sticks for hitting the can. and then ice and ice skates. This is of course an extremely crude reconstruction of the origin of ice hockey. but then it is not meant to be a historical account. It is. to use Rudyard Kipling’s felicitous locution. a "just so" story. You recall, "How the Elephant Got His Trunk" and so on.

Clearly I am not trying to be a sports historian with this crude fable. I am merely suggesting, not terribly imaginatively, how unsophisticated play can become sophisticated play, that is, game. 

Now let us consider a "just so" story that explains the origin of sport performances in contrast to sport \textit{games}. A member of an economically primitive culture is engaged in food gathering. He does this by diving into a tropical pool to catch fish. Time passes, the fishing pole and fishing nets are invented, and so it is no longer necessary to employ the skill of diving in order to secure food; indeed. it would be inefficient to do so. Yet the skill of diving into the water has come into existence. Shall our primitive fisherman. then, just quit diving, as a primitive farmer would presumably release his wife from pulling the plow when he got himself a horse? Well it is conceivable. though just barely, that the plow—pulling wife, upon being replaced by a horse, might respond that she was suffering a deprivation. After all. she had spent years acquiring the skill of plow—pulling and now it was all for naught. Perhaps she might be allowed to pull the plow after hours, when the horse wasn't using it. This is what may be called a \textit{not} Just So story. 

Let us then return to the now technologically unemployed fish diver, which is, or at least could be. a quite different kettle of \ldots ah \ldots fish. He, let us say in our plausible lust So fashion. really does want to continue to exercise the skill he has learned but which has become, as the British so delicately put it, redundant. And so he continues to dive. but not for fish because that is no longer the payoff. The payoff has changed from what was formerly instrumental to something else. And so he continues to dive because he enjoys exercising that skill and also, perhaps. because he enjoys displaying that skill to others. 

The foregoing suggests that games generate new skills and that performances refine (make more sophisticated) already existing skills. Thus hockey, for example. is not the refinement of an already existing skill In the first place. getting a hard rubber disc across a line drawn on ice is not an already existing goal. And if it were. then putting on skates, learning how to use them, and further burdening oneself with a stick to propel the disc across the line (instead of sensibly using your hand) would not be an already existing skill requiring refinement. No, games generate new skills by erecting artificial constraints just so those constraints can be overcome. 

But the case with our diver is very nearly the opposite of that. Since he is no longer diving for fish, he is liberated, as it were, from the constraint of diving at a certain time in a certain way in a certain place. He would not, for example, do back somersaults to catch fish, nor take special care to enter the water in as nearly a vertical posture as possible. We may conclude, therefore, that games generate \textit{new} skills by erecting \textit{artificial} constraints just so those constraints can be overcome. whereas performances eliminate natural constraints in order to refine old skills. There is just one thing wrong with this conclusion. It isn’t true. The triad of play, game, and sport is far too tricky to admit of such a tidy resolution.
 
The falsity of the resolution I have proposed can be established by constructing two more lust So stories. One will show that games can come to be by refining old skills every bit as much as performances can. and the other will show that performances can generate new skills every bit as much as games can. 

Once upon a time there was a diver who dived for fish for food. One day the pond into which he dived contained. unbeknown to him, a crocodile who was also fishing for food. Very soon the latter was in hot pursuit of the former. The diver, who was also, not surprisingly, a swimmer. swam away from the crocodile as fast as he could and was soon safely ashore and up a tree, filled not only with immense relief but also with a feeling of intense exhilaration at having outdistanced his aquatic competitor. The crocodile grumbled his way out of the pond to seek fresh waters and he never returned, so the pond was once again safe for unimpeded aquarian foraging. A happy state of affairs? Not quite. For the diver kept thinking back on the exhilaration he had felt at his swimming victory, and on the satisfaction he had derived from his competitive swimming skills. He pondered. Should he introduce more crocodiles into the pond in order to repeat his earlier victory? But suppose he lost? In that case the game would certainly not be worth it. Then the light dawned, and swimming races were invented. That is to say. a game came into being by refining an old skill.
 
Can you handle mother story? O.K. Once upon a time there was a diver who dived for fish for food. No, this isn’t the same story. The diver in this one is a different diver. And the fact that he dived for fish for food is irrelevant anyway. And besides that, there is no crocodile in the story.
 
No, there is just this person. actually. standing on a bank overhanging a pool of water. On previous occasions our subject had jumped. or sometimes simply flung himself any old way into the water for the fun of it, much as the baby splashed the water in its bath for the fun of it. But one day it came to him that he could also enter the water in a more interesting and demanding way. Why did this idea came to him? Boredom, I should think. which is the mother of play. \textit{Boredom is the mother of play}. I was going to add, "Remember, you heard it here first.‘ And you may have. But I didn't. I stole it from Kierkegaard. But I digress. Boredom prompted him to try entering the water by doing a back somersault. With practice, he became quite adept at this new skill and performed the dive for his own enjoyment and that of a number of onlookers who often gathered to watch the performance. That is to say, a performance came into being by creating a \textit{new} skill. 

So. Games can come from work (that is. instrumental activities, like outswimming a crocodile) as well as from play, and so can performances. Does this mean that there is no difference between them? Not at all. It is just that the difference does not lie where it initially appeared to lie. Even so, we have turned up, in our abortive effort to distinguish the two, two other distinctions. namely, the distinction between work-generated and play-generated sport and me distinction between the creation of new skills and the refinement of old skills, and these distinctions may prove useful in illuminating questions other than the one now at issue, namely, the difference between games and performances. 

So let us, as Aristotle so frequently used to say, make a new beginning. I have suggested that games are essentially refereed events and that performances are essentially judged events. This, I suggest, is the key to where the difference lies. In games, artificial barriers are erected just so they can be overcome by the use of rule-governed skills. Rules are the crux of games because it is the rules of any particular game that generate the skills appropriate to that game. Blocking in football is a skill required by, and thus generated by, the rules because the rules rule out, among other things, the use of machine guns by guards and tackles. That they are refereed calls attention to this fact, because referees see to it that the rules are followed and impose penalties when they are not. But, it might be asked, what about games that have no referees---a pick-up game of football, hockey, or whatever. The answer is that the players of such games, if they are sincere players, referee themselves. The rule-enforcement function is the same in both cases. 

Performances, on the other hand, are not rule-governed in that way at all. There are rules, to be sure, but not only are they \textit{not} the crux of performances, they usually take the form of applying to the participants \textit{outside} the arena of contention. I dislike mentioning steroids at this time, but the rule against their use would be the kind of rule typically at issue in the case of performances, what might be called perhaps a pre-event rule. But once a performance is under way. there are no rules, or scarcely any, that need enforcing. 

Now it may be objected that, contrary to what I have said, there clearly \textit{are} rules that must be followed while actually engaged in performative sports. For example, the gymnast must not falter or stumble after dismounting from the parallel bars. It is perfectly permissible to call such a requirement a rule, but it is quite clear, I should think. that such "rules" are entirely different from, say, the offside rules in football and hockey. The offside rule is what has come to be called, by me and many others. a constitutive rule, while the standard of a clean dismount from the parallel bars is a rule of skill, or a tactical rule, or at rule of practice. When I habitually fail to keep my eye on the ball in golf when driving from the tee. for example, I also habitually drive the ball halfway into the ground about two feet beyond the tee. But no one accuses me of cheating, just of incredible incompetence. The case is quite different, however, when I am discovered hand-carrying the ball across the green and dropping it into the cup.

Now, with the exception of what I have called pre-event rules (the steroid example), the rules to which the judges of performances address themselves are, I submit, rules of skill rather than constitutive rules. This suggests a way of distinguishing games from performances more likely to be successful than was my earlier attempt to distinguish them in terms of how their respective skills were \textit{generated}. For it is clear that games and performances both generate their skills, whether out of what I have loosely called work or play. Perhaps, then, the answer lies not in what the skills of games and performances are generated from, but in \textit{how} such skills are generated. And I shall forthwith suggest an answer that I do not believe to be false. Games do so, I suggest, by erecting barriers to be overcome, but performances do so by postulating ideals to be approximated. 

In games, \textit{rules}, to repeat the point, are the crux. Just these rules generate just these skills. In performances, ideals are the crux of the matter. Just these ideals generate just these skills. That is why it is possible to speak of a perfect performance, at least in principle, without fear of contradiction, whereas a perfectly played game, as I have tried to show elsewhere, seems to lead to a paradox. 

An objection at this point must be stated and met. In some games referees make what have come to be called "judgment calls." This may initially suggest to some that there is a hybrid sport, neither a game nor a performance, but a combination of both, what might be called a plotmance or I pame. I think not. Even though boxing referees and judges can differ in their calls, that is not like gymnastic judges holding up their numbered cards. For boxing officials, even when they differ, are not making their assessments against an ideal of boxing, if there is such a thing, but of one competitor directly against another, even if both are very far indeed from a boxing "ideal" performance. And a pitching umpire is not called upon to assess pitching skill against some standard of excellence: he must simply decide whether each pitch does or does not cross the plate within the strike zone. Such judgment calls not only are \textit{not} the crux of such games in which they are found but, much more important, they are thought of as something very close to being necessary or unavoidable evils. Referees make judgment calls not as to overall performance, and indeed not with respect to performance skills at all, but with respect to \textit{events}. \textit{Did} the tennis ball land within the service court? \textit{Was} the dropped ball a fumble according to the rules? Did boxer x strike more blows, and more crucial ones, than his opponent? \textit{Did} and \textit{where did} the saber strike the opponent? That such referential judgments are regarded as less than the best way to decide issues such as these is evidenced by the introduction of technical scoring devices in fencing to \textit{replace} such judgments calls, and of televised playbacks in football. The judgment calls of referees are not judgments about degree of approximation to an ideal but only about what happened under what circumstances. 

\subsection{Sports and Play}

Let me begin by making the bald assertion that not all sports are play. That may seem to most of you to be an obvious point, and therefore one scarcely worth making. Perhaps I can make it more interesting by adding to it an additional claim, to wit, that the events of the recent Olympics and, I believe, of the Olympic Games since their beginning in the mists of Greek history, are not and were not play. Let us return, as I promised we would early in these remarks, to the distinction between what I called primitive and what I called sophisticated play. We have seen, or at least I have seen and hope that you have as well, that primitive play can, though it need not, turn into sophisticated play either in the form of games or of performances, and that these two types of event that figure in the Olympics, and in what are called amateur sports generally, also figure in what are called professional sports. Let us assume, but only for the moment, that the Olympic Games fall within the category of amateur sports. We may then display the relations between play, game, and sport by means of three overlapping circles [figure 3.1]. 

The portion of the circle labeled Play which does not overlap either Game or Sport, and which I have numbered 1. represents primitive play. The area where only Play and Game overlap is the game instance of sophisticated play. Area 2. Area 3 overlaps neither Play nor Sport. and so although we will find games there. such games are neither athletic events nor instances of sophisticated play. What are they? For the moment we shall call the events in Ara 3 professional nonathletic games. such events as professional bridge and poker. Area 4. where only Play and Sport overlap. but not Game, is where amateur performances are located. Area 5 contains amateur games. Area 6 is where professional athletics are found; it is the counter-pan of Area 3. where we found professional sport athletic events Area 7. the area of Sports that are neither Games not Play. includes. most significantly for the present analysis. professional athletic performances. for the two types of event that appear in our paradigm of Olympic Games have already been assigned to Areas 4 and 5. Whether Area 7 contains other things as well, that is, other sports that are neither play nor games. I leave an open question.

At this point a question may arise. If, as I have argued. games and performances are instances of sophisticated play, how is it possible for there to be games and performances that are not play, as in 3, 6, and 7? The answer is that when games become instruments for external purposes (most obviously for acquiring money, as in the form of salaries drawn by players in the NHL, CFL. and so on), then these games, just like their players, lose their amateur standing. And that simply means, consistent with the terminology I have been using throughout. that such games are not played primarily out of love of the game but out of love of what the game can produce, whether playing it is loved or not. Accordingly, I would like to turn now to my concluding section. 

\subsection{Amateur Versus Non—Amateur Standing}

In these remarks, l have employed several different terms in referring to a pair of opposites generally acknowledged to be of central importance in a discussion of sports and games. They are. work versus play, serious versus nonserious. instrumental versus noninstrumental. and finally. professional versus amateur. These different verbalizations were intended to refer pretty much to the same pair of opposed conditions. There is a danger of being misled. however. particularly with respect to the distinction between professional and amateur. One is inclined, at least at first. to think of CF]. players, for example. as professionals and university players as amateurs—at least in Canada. In the States the division between pro and college football is somewhat less clear. to be sure. 

Now with respect to games. and performances as well. the word amateur. it scents to me. is a better word to use for one side of the distinction. for it does convey the notion of play rather than work. of the nonserious rather than the serious. and of the noninstrumental rather than the instrumental. But the word professional is. taken in its usual sense. too narrow to convey what the words work. serious. and instrumental convey. Still. while the word amateur in its literal meaning (one who does it for the love of it) does convey what is wanted. the connotation of amateur strongly suggests that it is simply the opposite of professional. I would like therefore to dispense with both words, and substitute for the phrase “amateur event or activity” the phrase ‘autotelic event or activity." that is. an event or activity valued for itself. And I would like to substitute for “professional” the expression ‘instrumental activity or event.” that is, an event or activity valued not only or even primarily for it~ self but for some further payoff that the event or activity is expected to provide. And the salary of a professional athlete is. to be sure, a clear example of such an instrumental payoff. But there are other payoffs as well, and those who play or game for them, though not called professionals. are engaged, I shall argue. in instrumental rather than in autotelic activities every bit as much as are professionals. 

I should like to put it to you that the games and performances of the Olympics may not be instances of sophisticated play. that is, that they may not be autotelic activities but instrumental activities. The point I want to make is a quite different one from the perennial. or rather quadrennial. point that the Soviet Union. East Germany. and others enter paid professionals into what is supposed to be an amateur undertaking. My point is. rather. that the Olympics. even if all the competitors were amateurs in the ordinary understanding of that word. would not provide us with good examples of games and performances as play. which is to say that such events would not fall within Areas 4 and 5 of our diagram. as one might suppose. but in Areas 6 and 7.
 
The reason l say this is because. for participants in Olympic events. playing the game is not the primary payoff‘ the players are seeking Getting the gold. either for themselves or for their homelands, is the primary payoff'{—or if not the gold. then the silver or bronze. That is why professionalism has crept into the Olympics. and it is also why steroids have crept in. But paid players and chemicals are not in themselves what render Olympic events instrumental rather than autotelic activities. Even in the absence of paid players or chemicals. 1 would still want to suggest that they are instrumental rather than autotelic events. For in the Olympics there is a kind of compulsion to win that is absent from a friendly game of tennis. or a pick-up game of baseball or hockey. l am suggesting that acting under such a compulsion. rather than the desire to win simply because winning defines the activity one is undertaking. is what turns a game that could be play into something that is not play. 

lnanold New Yorkercanooma portlyand agitated man dressed in the latest golf toggery is seen speaking angrily to his golfing partner. The caption reads. "Stop saying it's just a game! Godammit. it's not just a game!‘ And he is quite right. For him. golf is not play. and so it is not. therefore. just a game.


\bibliography{sample-handout}
\bibliographystyle{plainnat}

\end{document}
